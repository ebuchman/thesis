
\begin{titlepage}
    \begin{center}
        \vspace*{1cm}
        
        \textbf{\large{Tendermint: Byzantine Fault Tolerance in the Age of Blockchains}}\\
        
        \vspace{1 cm}

        \textbf{by} \\
        \vspace{1 cm}
        \textbf{Ethan Buchman}
        
        \vfill
        
        A Thesis \\
	presented to \\
	The University of Guelph 

        \vspace{0.8cm}

	In partial fulfilment of requirements \\
	for the degree of \\
	Master of Applied Science \\
       	in \\
	Engineering Systems and Computing

	\vspace{0.8cm}
	Guelph, Ontario, Canada

	\vspace{0.8cm}
	\copyright Ethan Buchman, May, 2016
    \end{center}
\end{titlepage}

\clearpage

\thispagestyle{plain}
\begin{center}
    \textbf{ABSTRACT} \\ 
    \vspace{0.6cm}
    \textbf{TENDERMINT: BYZANTINE FAULT TOLERANCE IN THE AGE OF BLOCKCHAINS}
\end{center}
    
\vspace{0.6cm}
\begin{tabular}{ p{0.5\textwidth} p{0.5\textwidth} }
Ethan Buchman & Advisor:\\ 
University of Guelph, 2016 & Professor Graham Taylor
\end{tabular}


\vspace{0.9cm}
Tendermint is a new protocol for ordering events in a distributed network under adversarial conditions.
More commonly known as consensus or atomic broadcast, the problem has attracted significant attention
recently due to the widespread success of digital currencies, such as Bitcoin and Ethereum,
which successfully solve the problem in public settings without a central authority.
Tendermint modernizes classic academic work on the subject to provide a secure consensus protocol with 
accountability guarantees, as well as an interface for building arbitrary applications above the consensus.
Tendermint is high performance, achieving thousands of transactions per second on dozens of nodes distributed around the globe,
with latencies of about one second, and performance degrading moderately in the face of adversarial attacks.

\clearpage

\thispagestyle{plain}
\par\vspace*{.35\textheight}{\centering Dedicated to Theda. \par}

\chapter*{Preface}
The structure and presentation of this thesis was much inspired by Diego Ongaro's 2014 Doctoral Dissertation, 
``Consensus: Bridging Theory and Practice'', wherein he specifies and evaluates the Raft consensus algorithm.

Much of the work done in this thesis was done in collaboration with Jae Kwon, who initiated the Tendermint project.
Please see the Github repository, at https://github.com/tendermint/tendermint, for a more direct account of contributions to the codebase.


\chapter*{Acknowledgments}
I learned early in life from Tony Montana that a man has only two things in this world, his word and his balls, and he should break em for nobody.
This thesis would not have been completed if I had not given my word to certain people that I would complete it.
These include my family, in particular my parents, grandparents, and great uncle Paul, and my primary adviser, Graham,
who has, for one reason or another, permitted me a practically abusive amount of flexibility to pursue the topic of my choosing.
Thanks Graham.

Were it not for another set of individuals, this thesis would probably have been about machine learning.
These include Vlad Zamfir, with whom I have experienced countless moments of discovery and insight;
My previous employer and favorite company, Eris Industries, and especially their CEO and COO, Casey Kuhlman and Preston Byrne,
for hiring me, mentoring me, and giving me such freedom to research and tinker and ultimately start my own company with technology they helped fund;
Jae Kwon, for his direct mentorship in consensus science and programming, for being a great collaborator, and for being the core founder and CEO at Tendermint;
Lucius Meredith, for mentoring me in the process calculi;
Zach Ramsay, for being, for all intents and purposes, my heterosexual husband;
and of course, Satoshi Nakamoto, whomever you are, for sending me down this damned rabbit hole in the first place.

There are of course many other people who have influenced my life during the course of this graduate degree;
you know who you are, and I thank you for being that person and for all you've done for me.

\pagenumbering{roman}
\tableofcontents
\listoffigures
\listoftables
\pagenumbering{arabic}

