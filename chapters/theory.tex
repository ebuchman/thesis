\chapter{Theory}

\section{Consensus and Atomic Broadcast}
The problem has been pitched as consensus or atomic broadcast (ABC).
Consensus commits a value; ABC orders transactions.
Can show they are the same \cite{chandra1996unreliable}
We show they are the same with generalized process calculus forms of each and a bi-simualtion between them.
Atomic broadcast is the more natural form for real systems.

Note the pi calculus doesn't allow a strictly composable encoding of broadcast \cite{ene1999expressiveness},
but we don't need it, since in practice each node has a network stack/kernel that manages broadcasts.
Further, we really do want point-to-point, rather than broadcast,
because we want connections to be encrypted on a per-connection basis,
though group-encrypted broadcast primitives would be an interesting pursuit.

Reliable broadcast (RBC) is a broadcast primitive satisfying

\begin{itemize}
\item validity - if a correct process broadcasts m, it eventually delivers m
\item agreement - if a correct process delivers m, all correct processes eventually deliver m
\item integrity - m is only delivered once, and only if broadcast by its sender
\end{itemize}

We model RBC as a pi-calculus process, 
$rbc(\hat{r}, \hat{d}) = (\nu \hat{x}) \prod_i rb_i(r_i, d_i, \hat{x})$,
where $rbc_i$ is the instance of RBC running on node $i$, 
$\hat{r}$ are input channels, with one for each node, 
on which new requests from clients can be received, 
$d_i$ are $delivery$ channels, on which a node outputs RBC-delivered values,
and $\hat{x}$ are some shared variables.

We can state the properties in a temporal henessy-milner logic with regular expressions: %fixed-point operators:
\begin{itemize}
\item validity - $ \forall m$, and correct $i$, $rbc |= [ r_i?(m) ] . < * > d_i!(m) $
\item agreement - $ \forall m$, and correct $i$, $rbc |= [ d_i!(m) ] . \wedge_{j \neq i} ( [ * ] d_j!(m) )$
\item integrity - $ \forall m$, and correct $i$, $rbc |= [ d_i!(m) ] . [ * ] . < d_i!(m) > ff $, and only if broadcast by its sender ...
\end{itemize}

Let us now model atomic broadcast ABC after RBC, as 
$abc(\hat{r}, \hat{d}) = (\nu \hat{x}) \prod_i abc_i(r_i, d_i, \hat{x})$,
with the same properties as $rbc$, but with the addition of \emph{total order},
\begin{itemize}
\item total order - if correct processes p and q deliver m and m', then p delivers m before m' iff q delivers m before m'
\end{itemize}


That is, ABC is identical to RBC, with the added constraint that reads off of any $d_i$ 
must return the same values in the same order.

We can model consensus similarly, as 
$cns(\hat{r}, \hat{d}) = (\nu \hat{x}) \prod_i cns_i(r_i, d_i, \hat{x})$,

with the following properties

\begin{itemize}
\item termination - every correct process eventually decides
\item integrity - every correct process decides at most once
\item agreement - if one correct process decides $v1$ and another decides $v2$, then $v1=v2$
\item validity - if a correct process decides $v$, at least one process proposed $v$
\end{itemize}

Note that the forms of consensus and ABC are identical (save some function names),
with the major difference in the properties relating to the fact that consenus
manages only one value, while atomic broadcast may handle many.

To show an equivalence between ABC and consensus,
we create a process context for each,
yielding $ C_{CNS}[ abc_i ] $ and $ C_{ABC}[ cns_i ] $ 
where we intend to show that 
$ C_{CNS}[ abc_i ] \sim cns_i $ and $ C_{ABC}[ cns_i ] \sim abc_i $ for
some weak bisimulation $\sim$.

Intuitevely, consensus can be derrived from ABC by deciding the first value fired on $d_i$,
while ABC can be derrived from consensus by running the consensus protocol multiple times,
once for each value to be atomically broadcast.
Thus $ C_{CNS}[ ] $ is a context which restricts $d_i$, such that it is only read from once,
while $ C_{ABC}[ ] $ is a context which manages multiple instances of consensus, delivering on $d_i$ many times.


\section{Byzantine Failure Detectors}
Failure detectors (FDs), an abstraction of timeouts,
were introduced and used to solve consensus \cite{chandra1996unreliable}.
We introduce byzantine failure detectors (BFD), which track byzantine faults,
defining a form of \emph{accountability}.
Most previous byz algos dont focus on detection, just tolerance.

FDs can be formalized with the pi-calculus, 
and resulting consensus protocols subject to a matrix analysis \cite{nestmann2003modeling}.
We'd like a similar analysis, with a more general notion of justification.

Further, we'd like to show that justifications can be removed from the real-time
protocol and moved to a post-failure recovery mode protocol, under some weak network assumptions,
without compromising accountability.

Start by defining messages as consisting of three parts: indices, authenticators, data.
Indices are things like height number, round number, message type number, etc.
Authenticators are signatures and hashes.
Cant do BFT without authenticators (tho wtf about some of those papers ...)
Byzantine msgs are those with the same indices and authenticators, but different data.
Note this assumes deterministic authenticators, and implies that detection requires gossip.
We also want byzantine msgs to be those that are "unjustified".
Introduce "justification" rules which map $(AUTH, DATA)$ to $\{True, False\}$.

Also note how moving data/indices into auth using hashes can simplify protocols
(eg. the way linking to the previous block avoids subtle leader crash/recover scenarios).

\section{Probabilistic Solutions}
Consensus can be solved with FDs or with randomness.
Common coin gives probabilistic liveness, where randomness is over what value sent.
Bitcoin gives probabilistic safety, where randomness is over when value sent.
There seems to be a duality here, common coin being like $\wedge$ and bitcoin like $\vee$.
How to reflect in stochastic-pi calc logic.

Bitcoin makes synchrony assumption that network latency is much less than block time,
allowing it to give strong (economic/probabilistic) serializability guarantees.
GHOST weakens the synchrony assumption by using additional network information to inform fork choice.
Is the asynchronous generalization of GHOST something like casper?
How does the move from PoW to PoS complement that from synchrony to asynchrony?

\section{Cryptography vs. Entropy}
\emph{Trust} as expected mutual information.
\emph{Correct-trust} as mutual information where you can see when it fails (crypto).
Show that correct-trust increases possible trust.

Digital signatures.
Merkle trees, erasure codes for broadcast.
Hash-chain links to simplify proposer logic.
Reduce complexity of network protocols by moving elements from data to authenticators.

\section{Economics}
Suppose the consensus system is probabilistic, ala some stochastic process calculus.
Economics are a way to parameterize the Comm rates of the calculus,
such that the param values may change, subject to some constraints 
(eg. the avg value over time is constant, etc.).
The point of the system is to be valuable,
and have this value be contributed back to the processes as wealth.
Economics makes the system reflexive, in the sense that,
given finite critical resources and a driving energy source,
the system must increase its efficiency (ie. innovate, build wealth, etc),
to maintain liveness during growth.
Integration with food systems, be an organism, etc.

\section{Residence Times}
Drawing inspiration from ecology and biophysics, 
where its been suggested that residence time of energy in a non-equilibrium system is a 
measure of its organizational complexity.

Consider a network of processes in such a light.
Energy input is receipt of a msg. 
Causes a tree of execution. 
Residence time is (eg.) time until all branches of the tree either communicate with other trees or halt.
Here, txs are the inputs (ie. they should pay fees!).
Another energy input is eg. POW - can be measured as a packet of energy arriving as a new block.
Without inflation, packets arrive and are immediately released as heat, minus what is paid in fees,
which hang around as a balance and prolong the residence time.
The inflation increases the residence time, but is clearly unsustainable - distribution mechanisms are important tho!
Alternatively, in POS, packets come in as security deposits, which sit around for a long time ...
