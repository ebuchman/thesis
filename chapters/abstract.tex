
\thispagestyle{plain}
\begin{center}
    \textbf{ABSTRACT} \\ 
    \vspace{0.6cm}
    \textbf{TENDERMINT: BYZANTINE FAULT TOLERANCE IN THE AGE OF BLOCKCHAINS}
\end{center}
    
\vspace{0.6cm}
\begin{tabular}{ p{0.5\textwidth} p{0.5\textwidth} }
Ethan Buchman & Advisor:\\ 
University of Guelph, 2016 & Professor Graham Taylor
\end{tabular}


\vspace{0.9cm}
Tendermint is a new protocol for ordering events in a distributed network under adversarial conditions.
More commonly known as consensus or atomic broadcast, the problem has attracted significant attention
recently due to the widespread success of digital currencies, such as Bitcoin and Ethereum,
which successfully solve the problem in public settings without a central authority.
Tendermint modernizes classic academic work on the subject to provide a secure consensus protocol with 
accountability guarantees, as well as an interface for building arbitrary applications above the consensus.
Tendermint is high performance, achieving thousands of transactions per second on dozens of nodes distributed around the globe,
with latencies of about one second, and performance degrading moderately in the face of adversarial attacks.

\clearpage

