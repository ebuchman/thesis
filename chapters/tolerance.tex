\chapter{Fault Tolerance}
\label{ch:tolerance}

Tendermint is designed as a Byzantine fault tolerant state-machine replication algorithm.
It gaurantees safety so long as less than a third of validators are Byzantine, 
and gaurantees liveness similarly, so long as network messages are eventually delivered,
with weak assumptions about network synchrony for gossipping proposals.
In this section, we evaluate Tendermint's fault tolerance empirically by injecting 
crash faults, Byzantine faults, and arbitrary network delays.
The goal is to show that the implementation of Tendermint consensus does not compromise safety in the event of such failures,
that it suffers minimum performance impact, and that it is quick to recover.

\section{Crash Failures}

To evaluate the response to crash failures, 
block commit times were recorded as up to one third of the validators are crashed,
and as they are brought back online. 

\section{Byzantine Failures}

To evaluate the response to crash failures, 
block commit times were recorded as up to one third of the validators behave arbitrarily.
Since its infeasible to capture every instance of arbitrary behaviour,
a few implementations are provided which cover some important cases, namely:
network fuzzing, double signing proposals or votes, and violating locking rules.

\section{Related Work}

