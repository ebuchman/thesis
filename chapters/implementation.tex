\chapter{Implementation}
\label{ch:implementation}

The reference implementation of Tendermint is written in Golang \cite{golang} and hosted at https://github.com/tendermint/tendermint.
Golang is a C-like language with a rich standard library, concurrency primitives for light-weight massively concurrent executions,
and a development environment optimized for simplicity and efficiency.

The code uses a number of packages which are modular enough to be isolated as their own libraries.
These packages were written in the most part by Jae Kwon, with bug fixes, tests, and the occasional feature contributed by the author.
The most important of these packages are described in the following sub-sections.

\section{Binary Serialization}

Tendermint uses a binary serialization algorithm optimized for simplicity and determinism.
It supports all integer types (including varints, which are encoded with a one-byte length prefix),
strings, byte arrays, and time (unix time with millisecond precision).
It also supports arrays of any type and structs (encoded as a list of ordered values, ignoring keys).
It is somewhat inspired by Go's type system, especially its use of interface types, 
which can be implemented as one of many concrete types.
Interfaces can be registered and each concrete implementation given a leading type-byte in its encoding.

See github.com/tendermint/go-wire for more details.

\section{Cryptography}

Consensus algorithms such as tendermint use three primary cryptographic primitives: digital signatures, hash functions, and authenticated encryption.
While many implementations exist of these primitives, 
choosing a cryptography library for enterprise software is no trivial task, given especially the profound insecurity of the world's most used security library 
(i.e. OpenSSL \cite{openssl_insecurity}).

Contributing to the insecurity of cryptographic systems is the potential deliberate undermining of their security properties by government agencies 
such as the NSA, who, in collaboration with the NIST, have designed and standardized many of the most popular cryptographic algorithms in use today. 
Given the apparent unlawfulness of such agencies, as made evident, for instance, by Edward Snowden \cite{snowden}, 
many in the cryptography community prefer to use algorithms designed in an open, academic environment.
Tendermint, similarly, uses only such algorithms.

Tendermint uses RIPEMD160 as its cryptographic hash function, which produces 20-byte outputs. 
It is used in the Merkle trees of transactions and validator signatures, and for computing the block hash.
Go provides an implementation its extended library. RIPEMD160 is also used as one of two hashing functions by Bitcoin in the derivation of addresses from public keys.

As it's digital signature scheme, Tendermint uses Schnorr signatures over the ED25519 elliptic curve. 
ED25519 was designed in the open by Dan Bernstein \cite{ed25519}, with the intention of being high performance and easy to implement without introducing vulnerabilities.
Bernstein also introduced NaCl, a high level library for doing authenticated encryption that uses the ED25519 curve. Tendermint uses the implemention provided by Go in its extended library.

\section{Merkle Hash Tree}

Merkle trees function much like other tree based data-structures, 
with the additional feature that it is possible to produce a proof of membership of a key in the tree that is logarithmic in the size of the tree.
This is done by recursively concatenating and hashing keys in pairs until only a single hash is left, the root hash of the tree.
For any leaf in the tree, a trail of hashes leading from it to the root serves as proof of its membership \ref{fig:Merkletree}.
This makes Merkle trees particularly useful for p2p file-sharing applications, where pieces of a large file can be verified as belonging to the file without
having all the pieces. Tendermint uses this mechanism to gossip block parts on the network, where the root hash is included in the block proposal.

Tendermint also provides a self-balancing, Merklized binary tree, modeled after the AVL tree \cite{avl}, as a TMSP service called Merkleeyes.
The IAVL tree can be used for storing state of dynamic size, allowing lookups, inserts, and removals in logarithmic time.

\section{RPC}

Tendermint exposes HTTP API's for querying the blockchain, network information, and consensus state, and for broadcasting transactions.
The same API is available via three methods: GET requests using URI encoded parameters, POST requests using the JSONRPC standard \cite{jsonrpc}, 
and websockets using the JSONRPC standard. Websockets are the preferred method for high transaction throughput, 
and are necessary for receiving events.


\section{P2P Networking}

The P2P subprotocols used by Tendermint is described more fully in Chapter \ref{ch:subprotocols}.

\section{Reactors}

The Tendermint node is composed of multiple concurrent reactors, 
each managing a state machine sending and receiving messages to peers over the network, as described in Chapter \ref{ch:subprotocols}.
Reactors synchronize by locking shared datastructures, but the points of synchronization are kept to a minimum,
so each reactor runs mostly concurrently with the others.

\subsection{Mempool}

The mempool reactor manages the mempool, 
which caches transactions before they are packed in blocks and committed.
The mempool uses a subset of the application's state machine to check the validity of transactions.
Transactions are kept in a concurrent linked list structure, allowing safe writes and many concurrent reads.
New, valid transactions are added to the end of the list. 
A routine for each peer traverses the list sending each transaction to the peer, in order, only once.
The list is also scanned to collect transactions for a new proposal, 
and is updated every time a block is committed: committed transactions are removed, 
uncommitted transactions are re-run through CheckTx, and those that have become invalid are removed.

\subsection{Consensus}

The consensus reactor manages the consensus state machine, which handles proposals, voting, locking, 
and the actual committing of blocks.
The state machine is managed using a few persistent go-routines, 
which order received messages and enable them to be played back deterministically to debug the state.
These go-routines include the readLoop, for reading off the queue of received messages, 
and the timeoutLoop, for registering and triggering timeout events. 

Transitions in the consensus state machine are made either when a complete proposal and block are received,
or when more than two-thirds of either pre-votes or pre-commits have been received at a given round.
Transitions result in the broadcast of proposals, block data, or votes, which are queued on the internalReqQueue,
and processed by the readLoop in serial with messages received from peers.
This puts internal messages and peer messages on equal footing as far as being inputs to the consensus state machine, 
but allows internal messages to be processed faster, as they don't sit in the same queue as those from peers.

\subsection{Blockchain}

The blockchain reactor syncs the blockchain using a much faster technique than the consensus reactor.
Namely, validators request blocks of incrementing height until none of their peers have blocks of any higher height.
Blocks are collected in a blockpool and synced to the blockchain by a worker routine that periodically takes blocks from the pool
and validates them against the current chain.

Once the blockchain reactor finishes syncing up, it turns on the consensus reactor to take over.

\section{Conclusion}

The implementation of Tendermint in Golang takes advantage of the languages concurrency primitives, garbage collection, 
and type safety, to provide a clear, modular, easy to read code base with many reusable components. 
As will be seen in Chapter \ref{ch:performance}, the implementation obtains high performance and is robust to many different kinds of fault.
